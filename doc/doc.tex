\documentclass{article}
\usepackage{html}
\usepackage{hyperref}
\usepackage{tabularx}
\usepackage[top=0.75in,bottom=0.75in,left=1.25in,right=1.25in]{geometry}
\usepackage{array}
\usepackage{graphics}
\hypersetup{
   pdfinfo={ %
      Title={Z4800 Documentation} %
   }, %
   bookmarks=true,         % show bookmarks bar?
   unicode=false,          % non-Latin characters in Acrobat’s bookmarks
   pdftoolbar=true,        % show Acrobat’s toolbar?
   pdfmenubar=true,        % show Acrobat’s menu?
   pdffitwindow=true,      % window fit to page when opened
   pdfnewwindow=true,      % links in new window
   colorlinks=true,        % false: boxed links; true: colored links
   linkcolor=blue,         % color of internal links
   citecolor=green,        % color of links to bibliography
   filecolor=blue,         % color of file links
   urlcolor=blue           % color of external links
}

\newcommand{\intref}[2]{%
   \latexhtml{%
      \hyperlink{#1}{#2 (\S\ref*{#1}, p.\pageref*{#1})}%
   }%
   {%
      \htmlref{#2}{#1}%
   }%
}

\newcommand{\quotes}[1]{\latexhtml{``#1''}{"#1"}}
\newcommand{\tm}{\latexhtml{\texttrademark}{\begin{rawhtml}&\#8482;\end{rawhtml}}}
\newcommand{\rtm}{\latexhtml{\textregistered}{\begin{rawhtml}&reg;\end{rawhtml}}}

\title{Z4800 Platform Documentation}
\author{Will Simoneau}

\begin{document}
\maketitle

\begin{rawhtml}
<div id="author_hack"></div>
\end{rawhtml}

The Z4800 platform is an open-source (mostly GPLv2) project consisting of a
synthesizable soft-core CPU plus supporting hardware and software. The included
reference designs allow construction of a complete FPGA computer complemented
by a full-featured operating system capable of running many real-world
programs. This platform can be used for many purposes, particularly benchmark
studies; the processor core can be modified and has an extendable performance
counter system.

The Z4800 CPU itself is a clean-room design capable of executing some binaries
compiled for a variant of the MIPS\rtm{} R4000\tm{} processor. The main
differences are that it does not implement the 64-bit or floating-point
instructions, and that its native endianness is little-endian.  Aside from
these differences, most existing user-mode and kernel-mode software will work.

A fully functional Linux kernel is supported; source code for the modified
kernel is included. These modifications are mostly to support the peripherals
on the FPGA boards -- the Z4800 CPU functions enough like an R4K that the
existing Linux/mips R4K code can be used directly. SMP (multiprocessor) support
is included.

To download the complete Z4800 source code, run the following command. You will
need at least Git 1.5.3 because this project uses submodules.
\begin{verbatim}
git clone --recursive git://trogdor.ele.uri.edu/z4800.git
\end{verbatim}

A paper about this project,
\htmladdnormallink{%
      \emph{%
         An FPGA-based Multi-Core Platform for Testing and Analysis of
         Architectural Techniques%
      }%
   }%
   {http://www.ele.uri.edu/~simoneau/ispass2012.pdf}%
, was written for and accepted to the
\emph{%
   \htmladdnormallink{%
      2012 IEEE International Symposium on Performance Analysis of Systems
      and Software%
   }%
   {http://ispass.org/ispass2012}%
}.

\smallskip

\subsection{To-Do List}
In no particular order:
\begin{itemize}
\item Publish public documentation on the Z4800 CPU internals
\item Add screenshots to the documentation to show example output
\item Create and publish virtual machine images providing a canned development environment
\item Create and publish tarball providing a canned root-FS for target board (DE2-115, etc.)
\item Publish public version of benchmarking infrastructure, with accompanying documentation
\end{itemize}

\subsection{Legal Notice}
{
   \footnotesize
   MIPS and R4000 are trademarks or registered trademarks of MIPS Technologies,
   Inc. in the United States and other countries. The Z4800 project is neither
   licensed by nor endorsed in any way by MIPS Technologies, Inc. We only claim
   to have limited binary compatibility; we claim neither full compatibility
   nor full compliance with any MIPS\rtm{} products and/or specifications.
}

\subsection{Contacts}

The author of the Z4800 source code is Will Simoneau
(\htmladdnormallink{simoneau@ele.uri.edu}{mailto:simoneau@ele.uri.edu}). He may
be contacted for any technical questions.
The ISPASS 2012 paper was co-authored with \htmladdnormallink{Resit
Sendag}{http://www.ele.uri.edu/faculty/sendag/}
(\htmladdnormallink{sendag@ele.uri.edu}{mailto:sendag@ele.uri.edu}).

\section{Building \& Booting}

The most complicated part of using the Z4800 design is setting up a working
development environment for it. Many pieces of software need to work together
for the developer (\emph{you}) to accomplish anything useful.  Setting up such
an environment is no small task; doing so from scratch is extremely difficult
for a new developer. The goal of this section is to guide you through setting
everything up, to the point where you should be able to compile the reference
board designs and software from scratch and demonstrate them functioning on
real FPGA boards.

The environment was designed for Linux hosts. It should be possible to develop
on Windows using \htmladdnormallink{Cygwin}{http://www.cygwin.com}, but we
haven't tried constructing such an environment.  If your host absolutely must
run Windows, you will need to install Linux
(\htmladdnormallink{Debian}{http://www.debian.org}, etc.) inside of
\htmladdnormallink{VirtualBox}{http://www.virtualbox.org} and develop on the
VM.

\subsection{Altera Software}

You'll need \intref{sec:files}{Altera's Quartus II software}. Version 11.0sp1
should work. Run their Linux installers and get it set up.

\subsection{GNU mipsel Toolchain}
\label{subsec:toolchain}

The Z4800 CPU can run code compiled for a 32-bit only, little-endian variant of
the R4K. GNU binutils and GCC can provide the minimal cross-toolchain you'll
need to compile the bootloaders and Linux kernel. You'll need to compile these
yourself; this section covers how to do that. We will be targetting
\verb+mipsel-unknown-linux-gnu+.

You will need to install some dependencies to be able to compile these. On
Debian Stable (as of this writing, Debian \quotes{Squeeze}), you will need to
do the following:
\begin{verbatim}
# apt-get install build-essential libgmp3-dev libmpfr-dev libmpc-dev
\end{verbatim}

\subsubsection{binutils}

The first piece of the puzzle is a cross-binutils. Download the
\intref{sec:files}{binutils source} and get it installed. The final command
requires root access if you use the default installation paths, since it will
be installing to \verb+/usr/local+.
\begin{verbatim}
$ tar xvjf binutils-2.20.1.tar.bz2
$ cd binutils-2.20.1
$ ./configure --target=mipsel-unknown-linux-gnu
$ make
$ sudo make install
\end{verbatim}

\subsubsection{GCC}

Now we need a cross-compiler, so get the \intref{sec:files}{GCC source}. We
don't need to cross-compile userland binaries so we don't need any external
headers, and just a basic C compiler will do. GCC does not like being built in
its own source directory; you need to create a separate build directory for it.

\begin{verbatim}
$ tar xvjf gcc-4.5.3.tar.bz2
$ cd gcc-4.5.3
$ mkdir build
$ cd build
$ ../configure --target=mipsel-unknown-linux-gnu --without-headers \
  --without-threads --enable-languages=c
$ make
$ sudo make install
\end{verbatim}

\subsection{CPU Boot Code}

At this point you should have a functional cross-compilation environment, and
should be able to successfully issue \verb+make+ in \verb+core/boot+ in the
Z4800 git tree. This will build the bootloader(s) which will be embedded into
the FPGA's on-chip ROMs. If this does not work, something is not set up
correctly.  Usually the problem is that your new binutils, GCC, or the Altera
NIOS tools (particularly \verb+elf2hex+) are not in your \$PATH.

\subsection{Generate SOPC Files}

Next, you need to open the Quartus project (example:
\verb+core/projects/de2_115_smp/prj_de2_115_smp.qpf+) in Quartus. Go to the
\quotes{Tools} menu and run \quotes{SOPC Builder}. (Or run \verb+sopc_builder+
from the project's directory.)

The first time you use SOPC Builder with any Z4800 project, you will need to
add the right directory to the IP search path. In SOPC Builder, go to
\quotes{Tools}, \quotes{Options}, select \quotes{IP Search Path}, and add the
\verb+core/hdl+ directory (which is a subdirectory of wherever you left the
checked-out git repository). This should result in \verb+core/hdl/**/*+ being
searched, which should pick up all the custom SOPC Builder modules.

Now you can open the SOPC Builder system file for the project (example:
\verb+sys_de2_115_smp.sopc+). If everything goes well, no errors should be
listed. If that's the case, hit Generate and wait until it finishes
successfully. This can take a couple minutes depending on your workstation's
hardware.

\subsection{Compile Project in Quartus}

Now that all the relevant files are generated, you can compile the Quartus
project itself. Click \quotes{Processing}, \quotes{Start Complation}. If there
are no errors, this will take from several minutes to a few hours, depending on
your workstation's hardware. (You could also do this without using the Quartus
GUI: \verb+quartus_sh --flow compile project_name+ from inside the project's
directory.)

When compilation is complete, you will have a .sof file in the project
directory (example:
\verb+core/projects/de2_115_smp/prj_de2_115_smp_time_limited.sof+). This is the
bitstream image of the completed design, ready to be used to program the FPGA.

\subsection{Kernel, OS, and NFS-root Hints}

If you got this far successfully, the next part you need to build is the
target's Linux kernel. An example kernel .config is provided in the board's
directory (\verb+kernel.config+). You will probably need to adjust
\verb+CONFIG_CMDLINE+, as it hardcodes the network configuration for NFS-root.

You can build the kernel from the board directory using a command such as the
following, which will put object files (and the final \verb+vmlinux+ ELF
binary) into the \verb+kobj+ subdirectory.

\begin{verbatim}
$ mkdir kobj
$ make -C ../../kernel KCONFIG_CONFIG=../kernel.config ARCH=mips \
  CROSS_COMPILE=mipsel-unknown-linux-gnu- O=../projects/de2_115_smp/kobj \
  vmlinux
\end{verbatim}

\subsubsection{Target OS}

You will need to have a local NFS server prepared; I suggest serving a Debian
install. To set up this directory with a usable root filesystem, you can use
the \verb+debootstrap+ or \verb+cdebootstrap+ tools. The key to using these is
the \verb+--foreign+ option, which permits bootstrapping on architectures other
than that of the host machine. You will need to refer to the tools'
documentation and think about your network setup to finish this step. Make sure
your NFS server is properly exporting the directory which will become the
target's root filesystem.

It can be helpful to use \htmladdnormallink{QEMU}{http://www.qemu.org} to
emulate a somewhat generic MIPS machine and bootstrap the OS before attempting
to boot an FPGA; how you accomplish this is up to you. Either way, by doing it
yourself you will become familiar with how things work.

\subsubsection{Booting the Target}

The target board needs a way to load its kernel. It is possible to boot kernels
out of Flash memory (\verb+core/boot/boot_cfi.S+ is a minimal bootloader for
CFI Flash written in assembly), but that requires one to program the kernel
onto the Flash chip, creating a chicken-and-egg problem. Instead, you can use a
second board to control the target board and load its kernel. In this example
we will pair a DE2-70 debugger board with a DE2-115 target containing the Z4800
CPU(s).

You will need to make a custom ribbon cable to safely connect the two FPGA
boards. Start with a 40-pin ribbon with one connector at each end, about 8-10
inches long (a 40-pin single-disk IDE ribbon could work). \emph{You MUST remove
two wires from the ribbon: those connecting to pin 11 and pin 29.} These wires
connect to the FPGA's power rails; it is inadvisable to short the two boards'
power rails together. If you get this wrong, it may destroy the boards -- you
have been warned. Be very careful with the pinout, and refer to both the DE2-70
and DE2-115 manuals to be sure your connection is correct before powering
anything on.

Use this ribbon cable to connect the DE2-70 to the DE2-115; either of the
DE2-70's GPIO headers may be used. If you connect to GPIO0 (the left one), set
SW17 down; if you connect to GPIO1 set SW17 up (there is a bus \quotes{switch}
connecting the remote-DMA module to either header).

Now that the boards are connected, power on only the DE2-70, and program the
\verb+core/projects/de2_70_debugger/de2_70_debugger.sof+
file onto it. Then, power on the DE2-115 and program it. Press and hold KEY0
(reset) on both boards simultaneously to ensure the link state is sane; LEDG0
through LEDG3 on both boards should be OFF if it is working correctly.

Attach a straight-through DB9 serial cable to the DE2-70's serial port. Run
minicom to get a console for the DE2-70 (the serial parameters it uses are
115200 8N1).  Then, boot the debugger's OS by running
\verb+nios2-download -C N -g core/projects/de2_70_debugger/zImage+
where N is the JTAG cable number (see output from \verb+jtagconfig+).  If
everything went right, it should boot a uClinux-based OS and drop you to a
shell. Then, you need to run commands similar to the following (adjust as
needed to fit your network setup):

\begin{verbatim}
ifconfig eth0 hw ether 00:07:ed:0a:03:29
ifconfig eth0 192.168.1.80 netmask 255.255.255.0
passwd
\end{verbatim}

The DE2-70 runs an ssh daemon, so after you've set the root password you can
log into it remotely. Use \verb+scp+ to copy the following files onto the
DE2-70 board's ramdisk:

\begin{verbatim}
core/boot/boot_ram.elf
core/projects/de2_115_smp/kobj/vmlinux
\end{verbatim}

You will want to connect a second straight-through DB9 serial cable to the
DE2-115's serial port at this point so that you have access to the target's
local console. The serial parameters are the same as the debugger (115200 8N1).
You could just swap the cable over from the DE2-70 since you probably don't
need its console anymore, but it's convenient if your host has 2 serial ports
to just connect both.

Now you can use the \verb+z48d+ debugger program on the DE2-70 to connect to
the DE2-115 to load and boot the kernel. The commands to do this are:

\begin{verbatim}
$ z48d -n x
l
boot_ram.elf
l
vmlinux
R
r
Q
\end{verbatim}
(In order, these mean: start z48d, load, bootloader ELF image, load, kernel ELF
image, reset, run, quit)

If everything went perfectly, the Z4800 target should boot Linux with NFS-root.
Good luck!

\section{Files}
\label{sec:files}

The following table lists the packages this documentation was written against.
Newer versions of these packages may not completely match this documentation.
Caveat user.

{
   \latex{\smallskip \scriptsize \centering}
   \begin{tabularx}{\textwidth}{| l | l | X | l |}
      \hline
         File name &
         md5sum &
         Description &
         License \\
      \hline
         \htmladdnormallink{\verb+11.0sp1_quartus_linux.sh+}{https://www.altera.com/download/software/quartus-ii-se/11.0} &
         {\verb+916ea16caafbc314cf385d139295e0fe+} &
         Quartus 11.0sp1 Base &
         Proprietary \\
      \hline
         \htmladdnormallink{\verb+11.0sp1_devices1_linux.sh+}{https://www.altera.com/download/software/quartus-ii-se/11.0} &
         {\verb+6da9b0dc1efdfa4af8f7415db96002ec+} &
         Quartus 11.0sp1 Devices 1 &
         Proprietary \\
      \hline
         \htmladdnormallink{\verb+11.0sp1_devices2_linux.sh+}{https://www.altera.com/download/software/quartus-ii-se/11.0} &
         {\verb+5fc1730b6dc4a3a66963c815d1f1fbfe+} &
         Quartus 11.0sp1 Devices 2 &
         Proprietary \\
      \hline
         \htmladdnormallink{\verb+binutils-2.20.1.tar.bz2+}{http://ftp.gnu.org/gnu/binutils/binutils-2.20.1.tar.bz2} &
         {\verb+2b9dc8f2b7dbd5ec5992c6e29de0b764+} &
         GNU binutils &
         GNU GPL \\
      \hline
         \htmladdnormallink{\verb+gcc-4.5.3.tar.bz2+}{http://ftp.gnu.org/gnu/gcc/gcc-4.5.3/gcc-4.5.3.tar.bz2} &
         {\verb+8e0b5c12212e185f3e4383106bfa9cc6+} &
         GNU GCC &
         GNU GPL \\
      \hline
   \end{tabularx}
}


\end{document}
